\section{Fragen}
\subsection{Erste Frage}
\bfseries Warum und wann benutzt man limitierte Integrale? Was sind diese Limits bzw. Beschränkungen? \\
\mdseries Man benutzt limitierte Integrale zum Beispiel, wenn das Integral durch physische Größen begrenzt ist. In unserem Fall ist das Integral über die Geschwindigkeit, also die Position, begrenzt, da nur Positionen zwischen \textit{1} und \textit{-1} angefahren werden können. \\
Außerdem ist physisch nur eine Maximalgeschwindigkeit von \\
$v_{max} = \sqrt{\frac{20}{7}\cdot g\cdot sin(\alpha)\cdot l}$ \\
möglich. Dies sollte bei den Berechnungen berücksichtigt werden.

\subsection{Zweite Frage}
\bfseries Welche Zeitperiode macht Sinn für das Simulationsintervall? \\
\mdseries Das Zeitintervall sollte groß genug gewählt werden, damit deutlich wird, dass der Ball tatsächlich zum stehen kommt und sich nicht zu einem späteren Zeitpunkt wieder bewegt. Da der Ball nach fünf bis zehn Sekunden zum stehen kommt sollte das Zeitintervall entsprechend groß gewählt werden. Sinnvoll sind hier \textit{15} bis \textit{25} Sekunden.