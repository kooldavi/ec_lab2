\section{Bericht}

\subsection{Einleitung}
Ziel der Übung ist es mit Hilfe von Scilab bzw. Xcos eine Simulation des Balles auf einer geneigten Fläche mit Hilfe der Lösungen aus der ersten Übung aufzubauen.

\subsection{Durchführung}
Zuerst haben wir ein Grundgerüst der Simulation basierend auf Abbildung 1 aus der Aufgabenstellung gebaut. Im Superblock für das BOIP-Modell haben wir das Skript aus Übung 1 mit Hilfe von Schaltblöcken nachgebaut. Neben Konstanten, Blöcken für Multiplikation, Addition und Trigonomischen Funktionen verwendeten wir auch einen Saturationsblock zu Beginn der Schaltung da wir den Maximal- bzw. Minimalwinkel auf +40$^\circ$ / -40$^\circ$ Grad limitieren konnten. Der Grund dafür ist die Tatsache das egal wie groß oder klein der Winkel ist wir immer das Ziel haben diesen auf 0$^\circ$ zu stellen. Deswegen ist es irrelevant ob ein Winkel 40$^\circ$ groß ist oder 140$^\circ$.
Desweiteren nutzen wir einen IFTHEL-Block sowie einen Selektor um zu entscheiden ob $\beta$ größer als $\gamma$ ist und somit auswählen zu können ob $\alpha$ negativ oder positiv ausgegeben werden muss.

\subsection{Besonderheiten}
text

\subsection{Fazit}
text
